\documentclass[magyar]{article}
\usepackage[T1]{fontenc}
\usepackage[utf8]{inputenc}
\usepackage[magyar]{babel}
\usepackage[autostyle]{csquotes}
\DeclareQuoteAlias{dutch}{magyar}
\title{Petőfi Sándor ars poeticája}
\author{Horváth Dávid}
\selectlanguage{magyar}
\linespread{1.3}
\begin{document}
	\maketitle
	Az ars poetica költői hitvallást jelent, azaz a költő és a költészet feladatát. A következőkben 4 versen keresztül szeretném bemutatni, hogy Petőfi Sándor hogyan gondolkodott erről a témáról.
	
	Az 1844-ben megjelent A természet vadvirágai című költeményében választ az őt ért támadásokra, mivel a társadalom nehezen fogadta, hogy ösztönösen ír verseket. Ekkor elege lesz a kritikákból, és a művészt ösztönös zseninek állítja be, amivel szembemegy a klasszicizmus eszméivel, miszerint a művész szabályokat követ, és ezt mindenki meg tudja tanulni.
	
	Az 1846-ban megjelent Dalaim című versében önmagát jellemzi, csapongó hangulatokkal. A versben 1-1 hangulathoz 1-1 metaforát társít.
	
	A szintén 1846-ban megjelent Sors, nyiss nekem tért című művében új életcélokat keres, illetve talál. Ebben a versben bizonytalanság is megjelenik, mivel tenni szeretni valamit az emberiségért, ezzel egy világmegváltó szerepbe helyezkedik. Golgotával utal a krisztusi szerepre, ezzel az önfeláldozás is megjelenik. Erre a kereszt szóval is utal.
	
	Az 1847-ben megjelent A XIX. század költői című költeménye egy programvers, amiben magának, és költőtársainak kijelöli a feladatot, ami népvezér-szerep, ez jóval konkrétabb, mint az előző versben. Erre a \enquote{lángoszlop} metaforával utal, és itt is megjelenik egy bibliai hasonlat, Mózes. Ebben a versben a költészet toposzok segítségével kerül kifejezésre, ezek a következők:
	\begin{itemize}
		\item \enquote{húrok}
		\item \enquote{szent fa}
		\item \enquote{lant} - ez a költészet ősi szimbóluma.
	\end{itemize}
	Nem ismeri el egy érzelmeket megszólító költészet jogosságát, és gyávának nevezi azokat, akik nem szolgálják a nép ügyét. 
	
	A műben a jövőkép is megjelenik. A távoli jövőben általános boldogságot vizionál, az \enquote{és addig}, illetve a \enquote{majd} 3-szori ismétlésével nyomatékosítja, hogy ez nem fog hamar bekövetkezni. Az általános boldogság feltételeit az 5. versszakban tárgyalja, Vörösmarty Mihály Guttenberg albumba című költeményéhez hasonlóan. Itt a felvilágosodás eszméiről metaforákban beszél, ezek a következők:
	\begin{enumerate}
		\item bőség kosara - anyagi egyenlőség
		\item jog asztala - jogi egyenlőség
		\item szellem napvilága - tudás elterjedése
	\end{enumerate}
	Arról is beszél, hogy a feltételek megvalósulásához sok küzdelem és önfeláldozás szükséges, és ezért a nyugodt halál a jutalom.
	
	Ebben a versben is megjelenik a próféta-szerep, másnéven a VÁTESZ, mivel megjövendöli a jövendőt, és buzdítja az embereket ennek elérésére.
\end{document}
