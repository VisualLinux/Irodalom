\documentclass[]{article}
\usepackage[T1]{fontenc}
\usepackage[utf8]{inputenc}
\usepackage[magyar]{babel}
\usepackage{csquotes}
\setquotestyle[quotes]{austrian}
\title{Madách Imre: Az ember tragédiája}
\author{Horváth Dávid}
\linespread{1.3}
\begin{document}
	
	\maketitle
	
	Madách Imre Az ember tragédiája című műve egyik nemzeti drámánk a Bánk bán mellett. Míg a Bánk bán az idegen zsarnokkal való szembehelyezkedésről szól, addig Az ember tragédiája általános filozófiai kérdéseket vet fel, válaszokat kevésbé ad, így ez egy gondolkodtató mű. Ilyen kérdések többek között:
	\begin{itemize}
		\item Mi az emberi létezés értelme?
		\item Létezik-e szabad akarat?
		\item Van fejlődés?
		\item Mi az eszmék szerepe?
	\end{itemize}
	
	A mű az emberiség történetén vonul végig, azt 15 színre osztva. Az 1. 3 szín un. bibliai színek, ezek a 15-el keretet alkotnak.
	\begin{itemize}
		\item 1. szín:\\
			A mennyben játszódik, amikor az Úr befejezte a teremtést:
			\begin{displayquote}
				\enquote{Be van fejezve a nagy mű, igen A gép forog, az alkotó pihen.}
			\end{displayquote}
			Luciferen kívül minden angyal dicséri, és az Úr is elégedett a művel. Azonban nehezményezi, hogy Lucifer nem dicséri:
			\begin{displayquote}
				\enquote{Csak hódolat illet meg, nem bírálat.}
			\end{displayquote}
			Erre ő azzal válaszol, hogy:
			\begin{displayquote}
				\enquote{Nem adhatok mást, csak mi lényegem.}
			\end{displayquote}
			Eléri, hogy az Úr nekiadjon 2 fát, hogy kísértésbe vigye az emberpárt, az Úr azonban biztos abban, hogy ez nem fog sikerülni.
		\item 2. szín:\\
			Az előző színhez hasonlóan ez a paradicsomban játszódik, amiben Ádámnak és Évának csak 2 fa, a tudás, és az örökélet fája van megtiltva. Lucifer azonban ráveszi őket, hogy ne fogadjanak szót.
		\item 3. szín:\\
			Az Úr úgy bünteti meg az emberpárt, hogy kiűzi őket a paradicsomból, ők azonban tele vannak önbizalommal, és úgy gondolják, nincs szükségük segítségre. Lucifer igyekszik őket elbizonytalanítani:
			\begin{displayquote}
				\enquote{Keserves még egykor e tudásod, \\
			S tudatlanságért fogsz epekedni vissza.}
			\end{displayquote}
			Mivel Ádám szeretné látni a jövőjüket, Lucifer álmot bocsájt rájuk, és megmutatja nekik az emberiség történelmét, azzal a céllal, hogy bebizonyítsa nekik, hogy életük reménytelen harc lesz.
		\item 4. szín:\\
			Ettől a színtől kezdve a dráma álomban játszódik, emiatt Ádám, Éva, és Lucifer különböző személyek alakjában jelennek meg. Ebben a színben Ádám a fáraó, Éva rabszolgafeleség, Lucifer pedig miniszter. Ettől a színtől kezdve a legtöbb színben megjelenik egy eszme, amit a szerző megvizsgál, hogy boldoggá tette-e az emberiséget, ebben a színben ez a hatalom.
			
			A fáraónak ugyan megvan a hatalma azonban mégsem boldog mivel magányos. Önmaga képtelen észrevenni mások boldogtalanságát, ezt Éva szünteti meg. Ezután felszabadítja a rabszolgákat:
			\begin{displayquote}
				\enquote{Mert megtanítál a jajt hallanom\\
				Ne halljam többé\\
				Ím legyen szabad\\
				A szolganép.}
			\end{displayquote}
			Ettől Lucifer azonban óvja őket, ebben megjelenik Madách Imre népszemlélete:
			\begin{displayquote}
				\enquote{[...] a tömeg\\
				A végzet arra ítélt állata,\\
				Mely minden rendnek malmán húzni fog.\\
				Mert arra van teremtve.}
			\end{displayquote}
			Azaz, a nép minden rendet kiszolgál.
			\begin{displayquote}
				\enquote{[...] Már ma mentsd fel:\\
				Amit te eldobsz, ő meg nem nyeri,\\
				és új urat keres holnap magának.}
			\end{displayquote}
			\begin{displayquote}
				\enquote{[...] Mély tenger a nép: bármi napfény\\
				Sem hatja át tömét; sötét leend az.}
			\end{displayquote}
		\item 5. szín:\\
			Ennek a színnek az eszméje a demokrácia, ami a nép uralmát jelenti. Ebben a színben Ádám Miltiádész, Éva Miltiádész felesége, Lucifer pedig harcos.
			
			Ebben a színben keserű népszemlélet jelenik meg, mivel a nép visszarántja a sikereseket. A demokrácia működését vizsgálja a szerző, és arra jut, hogy ugyan az ember látszólag szabad, azonban ki van szolgáltatva a demagógoknak. Így a demokrácia eszménye eltorzul:
			\begin{displayquote}
				\enquote{Ha sárba hull a fényes, kicsinyessége,\\
					kárörömmel szemléli a pór [...]}
			\end{displayquote}
		 	Éva szerint a nyomor teszi őket kiszolgáltatottá:
		 	\begin{displayquote}
		 		\enquote{Én e gyáva népet meg nem átkozom\\
		 		Az nem hibás, annak természete\\
	 			Hogy a nyomor szolgává bélyegezze.}
		 	\end{displayquote}
	 	\item 6. szín:\\
	 		Ez egy eszme nélküli szín, Ádám Sergiolus, Éva Júlia, Lucifer Miló.
	 		
	 		Ebben a színben a szerző a hedonizmust vizsgálja, ami a hétköznapi élvezeteket (érzéki örömök mértéktelen élvezete) jelenti. Ez azonban undort, kielégületlenséget, és csömört hoz maga után. Az emberek vágynak valami után, ami értelmet vinne az életükbe, ekkor jelenik meg testvériség, kereszténység eszméje.
 		\item 7. szín:\\
 			
	\end{itemize} 
\end{document}
