\documentclass[]{article}
\usepackage[T1]{fontenc}
\usepackage[utf8]{inputenc}
\usepackage[magyar]{babel}
\title{Madách Imre: Az ember tragédiája}
\author{Horváth Dávid}
\linespread{1.3}
\begin{document}
	
	\maketitle
	
	Madách Imre Az ember tragédiája című műve egyik nemzeti drámánk a Bánk bán mellett. Míg a Bánk bán az idegen zsarnokkal való szembehelyezkedésről szól, addig Az ember tragédiája általános filozófiai kérdéseket vet fel, válaszokat kevésbé ad, így ez egy gondolkodtató mű. Ilyen kérdések többek között:
	\begin{itemize}
		\item Mi az emberi létezés értelme?
		\item Létezik-e szabad akarat?
		\item Van fejlődés?
		\item Mi az eszmék szerepe?
	\end{itemize}
	
	A mű az emberiség történetén vonul végig, azt 15 színre osztva. Az 1. 3 szín un. bibliai színek, ezek a 15-el keretet alkotnak.
	\begin{itemize}
		\item 1. szín:\\
			A mennyben játszódik, amikor az Úr befejezte a teremtést, Luciferen kívül minden angyal dicséri, és az Út is elégedett a művel. Azonban nehezményezi, hogy Lucifer nem dicséri. Eléri, hogy az Úr nekiadjon 2 fát, hogy kísértésbe vigye az emberpárt, az Úr azonban biztos abban, hogy ez nem fog sikerülni.
		\item 2. szín:\\
			Az előző színhez hasonlóan ez a paradicsomban játszódik, amiben Ádámnak és Évának csak 2 fa, a tudás, és az örökélet fája van megtiltva. Lucifer azonban ráveszi őket, hogy ne fogadjanak szót.
		\item 3. szín:\\
			Az Úr úgy bünteti meg az emberpárt, hogy kiűzi őket a paradicsomból, ők azonban tele vannak önbizalommal, és úgy gondolják, nincs szükségük segítségre. Lucifer igyekszik őket elbizonytalanítani. Mivel Ádám szeretné látni a jövőjüket, Lucifer álmot bocsájt rájuk, és megmutatja nekik az emberiség történelmét, azzal a céllal, hogy bebizonyítsa nekik, hogy életük reménytelen harc lesz.
	\end{itemize} 
\end{document}
