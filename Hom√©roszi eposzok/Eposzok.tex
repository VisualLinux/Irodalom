\documentclass[]{article}
\usepackage[T1]{fontenc}
\usepackage[utf8]{inputenc}
\usepackage[magyar]{babel}
\title{Homéroszi eposzok}
\author{Horváth Dávid}
\linespread{1.3}
\begin{document}
\maketitle

	A két eposz, az Íliász, és az Odüsszeia, valószínűsíthetően a kr.e. 8.sz.-ban keletkezett. Legfontosabb műfaji előzményei az un. genealógiai énekek lehettek, azonban nem ezek összefűzéséből állnak, ahogy azt korábban gondolták. Mindkettő nyelve műnyelv volt, és az újszerű versformája miatt sem lehet énekesen előadni, ebből arra következtethetünk, hogy tudatos alkotásokról van szó. Kétséges azonban, hogy a két eposzt ugyanaz a személy írta-e. A két eposz világszemlélete különbözik, emiatt az Odüsszeia alkotója szinte bizonyosan legalább egy emberöltővel később élt.\cite{TetelOdusszeia} A művek sajátosságai a következők:
	\begin{enumerate}
		\item Verselés:\\
		Időmértékes, versformájuk hexameter.
		\item Homéroszi jelzők:\\
		Pl.: görbeeszű, görbeíjú, görbekarmú, széphajú, jóhajú, magashajú.
		\item Ismétlődő szókapcsolatok
		\item Harcleírások
		\item Vendéglátás formái
		\item Sajátos költői műnyelv:\\
		Leginkább előre kész, a versbe illő formulákból épül fel.
		\item Szájhagyományozás:\\
		Az eposzokat a szájhagyományozó költészet hosszú korszaka előzte meg. Ennek szükségletei alakították ki a formulákból felépülő költői műnyelvet, az ismétlődő helyzetek elbeszélésének alapformáit. 
	\end{enumerate}
\cite{Sajatossagok}
	\section{Íliász}
		Az Íliász az egyik legjelentősebb ógörög eposz, melyet Homérosznak tulajdonítanak. A trójai háborúnak csak egy részéről szól, Akhilleusz haragjáról, és ennek következményeiről. A mű címe Trója régi névéből Ilion-ból származik. 15685 hexameterből áll, 24 énekre van tagolva.
		\subsection{Történeti háttér}
			A trójai mondakör eseményeit a 19. századig néphagyománynak tartották, amíg egy régész megtalálta Trója várost. A háborút, amely az Íliászban is megjelenik i.e. 1200 körül vívták Trója ellen a dél-görögországi Mükéné vezetésével a görögök.			\cite{WIliasz}
		\subsection{Történet}
			\begin{enumerate}
				\item Az invokációt követően in medias res kezdődik a történet a Trójai háború vége fele. Mivel a görög fővezér Átreidész Agamemnón elrabolta Khrüszészt, Phoibosz Apollón egyik papjának lányát így megharagudott a görögökre, ezért dögvészt bocsátott rájuk. Bár az apa felajánlott váltságdíjat, de ez a próbálkozása sikertelen maradt.
				
				Mikor Akhilleusz gyűlést rendezett, és megkérdezte a jóst, Kalhászt a járványról, azt a tanácsot adta, hogy ki kell adni a lányt. Emiatt enged Agamemnón, azonban megfenyegeti Akhilleuszt, aki megmérgesedik, és félrevonul. Emiatt anyja, Thetisz megkéri Zeuszt, hogy büntesse meg az akhájokat.
				
				\item Zeusz úgy bünteti meg Agamemnónt, hogy rossz tanácsot ad neki. Nesztór képében álmában a fejéhez lép, és azt mondja neki, hogy az egész sereget hívassa fegyverbe, mert most beveheti a várat. 
				
				Reggel összehívja a népet, és felszólítja őket, hogy menjenek haza, mivel ki szeretné próbálni a harci kedvüket. Pallasz Athéné azonban nem akarja, hogy a görögök ne bosszulják meg Tróját, így közbelép. Odüsszeuszt, és a többi görögöt arra biztatja, hogy folytassák a harcot.
				
				Eközben Irisz, az isteni hírnök figyelmezteti a trójaiakat, hogy támadás várható.
				
				\item Meneláosz szembekerül felesége csábítójával, Parissal, aki azonban menekül előle. Ezért megszidja bátyja, Hektór. Emiatt Parisz felajánlja, hogy megküzd Meneláosszal, aki elfogadja a kihívást.
				
				Priamosz is ott van a párbajnál. Parisz vetheti előbb a dárdáját, azonban nem sikerül sebet ejtenie ellenfelén. Meneláosz nem tud visszaütni, mivel Aphrodité ködbe burkolja, és elveszi Pariszt. Helené megveti a megfutamodottat.
				
				\item Meneláosz győzött, azonban a trójaiak megszegik az esküt, egyikük íjjal meglövi. Az istenek végképp eldöntötték, hogy Trójának kell pusztulnia, így Agamemnón a görögöket a harc folytatására buzdítja.
				
				\item A harcba beavatkoznak az istenek. Aphrodité a fiát, Aineiászt védi, azonban, mivel Diomédész által a kezén megsebesül, megfutamodik. Aineiászra ráront Diomédész, azonban Apollón beavatkozik, és kéri Arészt, a hadistent, hogy fékezze meg. Őt meg Pallasz Athéné segíti, és sebet ejt Arészon is, aki így apjához, Zeuszhoz rohan panaszkodni.
				
				\item Hektór a trójaiakat, akik már-már visszavonulnak, buzdítja, ő a várba megy, és Hekabét, az anyját megkéri, hogy a nőket gyűjtse Trójáért való imádkozásra Pallasz Athéné templomába. Elbúcsúzik Andromakhéttől, a feleségétől, illetve Asztüanaxtól, aki a fia.
				
				\item A görögök kisorsolják, hogy Aiász vív meg Hektórral. A küzdelem eredménye döntetlen lett. 
				
				\item Zeusz az istengyűlésen megtiltja, hogy az istenek beavatkozzanak a harcba. Amíg a görögök Akhilleusz híján vannak, hagyja győzni a trójaiakat.
				
				\item Menekülnek a görögök, mivel legjobb hőseikből sokat elvesztettek. Agamemnón attól tart, hogy sose veszik be a várat. Nesztór azt tanácsolja, hogy próbálja meg Akhilleuszt visszaszerezni. Agamemnón követséget küld a hőshöz, aki azonban nem tér vissza a harctérre.
				
				\item A görögök úgy határoznak, hogy meg kell tudni a trójaiak szándékait. Ezt Hektór is így gondolja, így kémet küld Dolón személyében, azonban elfogják, kivallatják, és megölik. Megtudja, hogy trákok érkeztek, akik a trójaiak szövetségesei, de már elaludtak. Diomédész Odüsszeusszal mészárlást rendez a táborukban, és elviszik a lovakat.
			\end{enumerate}
			\cite{WIliasz_en}
			\cite{Iliasz_tartalom}
\bibliography{Eposzok}
\bibliographystyle{ieeetr}
\end{document}
