\documentclass[]{article}
\usepackage[T1]{fontenc}
\usepackage[utf8]{inputenc}
\usepackage[magyar]{babel}
\title{Homéroszi eposzok}
\author{Horváth Dávid}
\linespread{1.3}
\begin{document}
\maketitle

	A két eposz, az Íliász, és az Odüsszeia, valószínűsíthetően a kr.e. 8.sz.-ban keletkezett. Legfontosabb műfaji előzményei az un. genealógiai énekek lehettek, azonban nem ezek összefűzéséből állnak, ahogy azt korábban gondolták. Mindkettő nyelve műnyelv volt, és az újszerű versformája miatt sem lehet énekesen előadni, ebből arra következtethetünk, hogy tudatos alkotásokról van szó. Kétséges azonban, hogy a két eposzt ugyanaz a személy írta-e. A két eposz világszemlélete különbözik, emiatt az Odüsszeia alkotója szinte bizonyosan legalább egy emberöltővel később élt.\cite{TetelOdusszeia} A művek sajátosságai a következők:
	\begin{enumerate}
		\item Verselés:\\
		Időmértékes, versformájuk hexameter.
		\item Homéroszi jelzők:\\
		Pl.: görbeeszű, görbeíjú, görbekarmú, széphajú, jóhajú, magashajú.
		\item Ismétlődő szókapcsolatok
		\item Harcleírások
		\item Vendéglátás formái
		\item Sajátos költői műnyelv:\\
		Leginkább előre kész, a versbe illő formulákból épül fel.
		\item Szájhagyományozás:\\
		Az eposzokat a szájhagyományozó költészet hosszú korszaka előzte meg. Ennek szükségletei alakították ki a formulákból felépülő költői műnyelvet, az ismétlődő helyzetek elbeszélésének alapformáit. 
	\end{enumerate}
\cite{Sajatossagok}
\bibliography{Eposzok}
\bibliographystyle{ieeetr}
\end{document}
