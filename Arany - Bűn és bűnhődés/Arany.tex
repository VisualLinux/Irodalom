\documentclass[]{article}
\usepackage[T1]{fontenc}
\usepackage[utf8]{inputenc}
\usepackage[magyar]{babel}
\title{A bűn és bűnhődés motívuma Arany János balladáiban}
\author{Horváth Dávid}
\linespread{1.3}
\begin{document}
	
	\maketitle
	
	A ballada eredetileg népköltészeti műfaj volt, pl.: Kádár Kata, Kőműves Kelemen. Arany János ebből hozta létre a műballadát, amit világirodalmi szintre emelt. A magyaron kívül forrásai még a skót és a székely népballadák. Jellemzője, hogy a 3 műnem határán van.
	\begin{center}
		
		\begin{tabular}{|c|c|c|}
			\hline
			Epika&Líra&Dráma\\\hline
			Történetet mesél el&Nyelvi kifejezésmód&Párbeszéd/monológ\\\hline
			&&Tragikum\\\hline
		\end{tabular}
	\end{center}
	Jellemzői még a következők:
	\begin{itemize}
		\item Balladai homály - kihagyások
		\item Lélektani beállítottság
		\item Bűn és bűnhődés
	\end{itemize}
\end{document}
