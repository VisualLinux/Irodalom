\documentclass[]{article}
\usepackage[T1]{fontenc}
\usepackage[utf8]{inputenc}
\usepackage[magyar]{babel}
\title{A bűn és bűnhődés motívuma Arany János balladáiban}
\author{Horváth Dávid}
\linespread{1.3}
\begin{document}
	
	\maketitle
	
	A ballada eredetileg népköltészeti műfaj volt, pl.: Kádár Kata, Kőműves Kelemen. Arany János ebből hozta létre a műballadát, amit világirodalmi szintre emelt. A magyaron kívül forrásai még a skót és a székely népballadák. Jellemzője, hogy a 3 műnem határán van.
	\begin{center}
		
		\begin{tabular}{|c|c|c|}
			\hline
			Epika&Líra&Dráma\\\hline
			Történetet mesél el&Nyelvi kifejezésmód&Párbeszéd/monológ\\\hline
			&&Tragikum\\\hline
		\end{tabular}
	\end{center}
	Jellemzői még a következők:
	\begin{itemize}
		\item Balladai homály - kihagyások
		\item Lélektani beállítottság
		\item Bűn és bűnhődés
	\end{itemize}

	Arany János egyik balladája, amelyikben megjelenik a bűn és bűnhődés motívuma az Ágnes asszony. A nagykőrösi balladák közé tartozik, és parasztballada. Az Arany János-i balladaköltészet szinte minden jellemzője megtalálható benne:
	\begin{itemize}
		\item Balladai homály
		\begin{itemize}
			\item Az elkövetett bűn csak a 11-12. versszakban válik nyilvánvalóvá.
			\item Véres lepel $\rightarrow$ bűntény
			\item Ágnes megőrülése
		\end{itemize}
		\item Lélektani érdeklődés:\\
		Nem az elkövetett bűn a lényeges, hanem, hogy egy egy ember hogyan viseli ennek a súlyát.
	\end{itemize}
	Körkörös szerkezete van, azaz a patakparti mosás egy visszatérő motívum. Ebben az azonos helyszín és cselekedet tér vissza az 1., a 20., és a 26. versszakban. Ennek a balladának az a fő mondanivalója, hogy egy elkövetett bűnért az igazi büntetés a bűntudat, amitől lehetetlen szabadulni.
\end{document}
