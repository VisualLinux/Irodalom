\documentclass[]{article}
\usepackage[T1]{fontenc}
\usepackage[utf8]{inputenc}
\usepackage[magyar]{babel}
\title{Petőfi Sándor ars poeticája}
\author{Horváth Dávid}
\linespread{1.3}
\begin{document}
	\maketitle
	Az ars poetica költői hitvallást jelent, azaz a költő és a költészet feladatát. A következőkben 4 versen keresztül szeretném bemutatni, hogy Petőfi Sándor hogyan gondolkodott erről a témáról.
	
	Az 1844-ben megjelent A természet vadvirágai című költeményében választ az őt ért támadásokra, mivel a társadalom nehezen fogadta, hogy ösztönösen ír verseket. Ekkor elege lesz a kritikákból, és a művészt ösztönös zseninek állítja be, amivel szembemegy a klasszicizmus eszméivel, miszerint a művész szabályokat követ, és ezt mindenki meg tudja tanulni.
	
	Az 1846-ban megjelent Dalaim című versében önmagát jellemzi, csapongó hangulatokkal. A versben 1-1 hangulathoz 1-1 metaforát társít.
	
	A szintén 1846-ban megjelent Sors, nyiss nekem tért című művében új életcélokat keres, illetve talál. Ebben a versben bizonytalanság is megjelenik, mivel tenni szeretni valamit az emberiségért, ezzel egy világmegváltó szerepbe helyezkedik. Golgotával utal a krisztusi szerepre, ezzel az önfeláldozás is megjelenik. Erre a kereszt szóval is utal.
	
	
\end{document}
