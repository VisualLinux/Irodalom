\documentclass[]{article}
\usepackage[T1]{fontenc}
\usepackage[utf8]{inputenc}
\usepackage[magyar]{babel}
\title{Rómeó és Júlia}
\author{Horváth Dávid}
\linespread{1.3}
\begin{document}
	
	\maketitle
	Shakespeare ezen műve valószínűsíthetően 1594 és 1596 között keletkezett, témája a szeretet és a gyűlölet. Itt a szerelem alatt az új típusú testi-lelki viszonyt kell érteni, mely a reneszánszban jelent meg először. Azaz hitvesi szerelemről van szó. Ez azonban  szembemegy a régi erkölcsökkel, mivel eddig a házasságkötés a családok közti megegyezésen alapult.
	
	A főszereplők nem átlagon felüliek, egy fiatal fiú, és egy 14 éves lány. A kettejükben lévő szenvedély teszi őket hősökké. A művön végigvonul a szerelem és a halál együttese A mű világában a szerelem halálra van ítélve.	\cite{erettsegi_com}
	
	A két híres veronai család, Capulet és Montague, régóta ellenségeskednek, azonban a két család gyermekei egymásba szeretnek. A fő konfliktus a szülők és gyerekek akaratának összeütközése. Rómeó és Júlia képviselik a reneszánsz eszméjét, akik csak a szokást vetették el.
		
	A kezdete akkor van, mikor Rómeó meglátogatja Capuleték bálját. A fordulópont Tybalt és Mercutius halála, amikért Rómeó a felelős, így menekülnie kell. Innentől két szálon futnak az események.
	
	Júlia látszólag beleegyezik, hogy Párissal összeházasodik, azonban megissza, amit Lőrinc baráttól kapott, amitől 48 órára elalszik. Ezzel az a célja, hogy a többiek elhiggyék, hogy halott. Rómeó azonban nem kapja meg Lőrinc barát levelét, és hírt kap arról, hogy Júliát eltemették, így öngyilkos lesz. Amikor Júlia ezt megtudja, ő is végez magával.
	\cite{12es}
	\bibliographystyle{ieeetr}
	\bibliography{RJ.bib}
\end{document}
